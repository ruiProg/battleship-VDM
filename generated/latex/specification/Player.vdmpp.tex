\begin{vdmpp}[breaklines=true]
class Player
types
 public string = seq of char;

instance variables
 protected name: string := [];
 protected wins: nat;
 protected losses: nat;
 protected ownBoard: [Board] := nil; --tabuleiro aonde coloca-se os proprios navios e regista-se as tentativas do inimigo
 protected enemyBoard: [Board] := nil; --tabuleiro das nossas tentativas de destoir a frota inimiga
 protected myShips: set of Board`CellContent := {}; -- navios colocados antes da ronda e meus navios durante as rondas (diminui)
 protected enemiesShips: set of Board`CellContent := {}; -- navios inimigos (aumenta)
 
 inv len name < 256;
 
operations
(*@
\label{Player:17}
@*)
 public Player: string ==> Player
  Player(nameArg) == (
   name := nameArg;
   wins := 0;
   losses := 0;
   return self
  );
   
(*@
\label{getName:25}
@*)
 pure public getName : () ==> string
  getName() == return name;
  
  -- reinicia jogo 
(*@
\label{addBoards:29}
@*)
 public addBoards : () ==> ()
  addBoards() == (
   ownBoard := new Board();
   enemyBoard := new Board();
   myShips := ownBoard.getShips();
  )
  pre ownBoard = nil and enemyBoard = nil;
  
  -- coloca navio no tabuleiro se este ainda n�o tiver sido colocado
(*@
\label{shipPlacement:38}
@*)
 public shipPlacement: Board`CellContent * char * nat1 * Board`Direction ==> ()
  shipPlacement(ship, colCh, line, dir) == (
   ownBoard.placeShip(ship, colCh, line, dir);
   myShips := myShips \ {ship}
  )
  pre ship in set myShips;
 
 -- retorna verdadeiro se todos os navios est�o colocados no tabuleiro
(*@
\label{allShipsPlaced:46}
@*)
 public allShipsPlaced: () ==> bool
  allShipsPlaced() == return card myShips = 0;
 
 -- inicia parte do jogo com o objetivo de destruir o navio inimigo 
(*@
\label{startRounds:50}
@*)
 public startRounds: () ==> ()
  startRounds () == (
   myShips := ownBoard.getShips();
   enemiesShips := {};
  );
 
 -- limpa tabuleiro e registo dos navios
(*@
\label{clearData:57}
@*)
 public clearData: () ==> ()
  clearData() == (
   ownBoard := nil;
   enemyBoard := nil;
   myShips := {};
   enemiesShips := {};
  );
 
 -- regista tentativa de afundan�o do advers�rio
(*@
\label{registerAttack:66}
@*)
 public registerAttack: char * nat1 ==> Board`CellContent
  registerAttack(colCh, line) == (
   dcl shipHit : Board`CellContent := ownBoard.registerMove(colCh, line);
   if ownBoard.countCellType(shipHit) = 0 then (
    myShips := myShips \ {shipHit};
    if card myShips = 0 then losses := losses + 1;
    return shipHit;
   )
   else if shipHit <> <Empty> then return <Hit>
   else return <Miss>;
  );
 
 -- regista a sua tentativa de afundar um navio advers�rio
(*@
\label{registerResult:79}
@*)
 public registerResult: Board`CellContent * char * nat1 ==> bool
  registerResult(code,colCh, line) == (
  if code = <Miss> then enemyBoard.setComponentCol(<Miss>,line,colCh)
  else(
   enemyBoard.setComponentCol(<Hit>,line,colCh);
    if code <> <Hit> then enemiesShips := enemiesShips union {code};
    if card enemiesShips = enemyBoard.getShipsCount() then(
     wins := wins + 1;
     return true;
    );
   );
   return false;
 );
 
--- print to console
 
 --disponibiliza informa��es do jogador
(*@
\label{printInfo:96}
@*)
 public printInfo : () ==> string
  printInfo() == return name ^ " (" ^ 
   VDMUtil`val2seq_of_char[nat](wins) ^ "-" ^ 
   VDMUtil`val2seq_of_char[nat](losses) ^ ")\n";
 
 -- disponibiliza o estado da coloca��o dos navios
(*@
\label{printPlacementStatus:102}
@*)
 public printPlacementStatus: () ==> string
  printPlacementStatus() == return "Fleet placement\nPlayer turn: " ^ name ^ "\n" ^
  "Ships to be placed: " ^ ownBoard.printRemainShips(myShips) ^ "\n\n" ^
  ownBoard.printBoard();
  
  --  disponibiliza o estado do jogo
(*@
\label{printGameStatus:108}
@*)
 public printGameStatus : () ==> string
  printGameStatus() == (
   dcl ret: string := "Player turn: " ^ name ^ "\nMy active ships: ";
   for all ship in set myShips do ret := ret ^ ownBoard.shipToString(ship) ^ "   ";
   ret := ret ^ "\nDestroyed enemies ships: ";
   for all ship in set enemiesShips do ret := ret ^ enemyBoard.shipToString(ship) ^ "   ";
   ret := ret ^ "\n\n                                 My ships \t\t\t\t\t\t\t\t\t                Enemy ships\n\n\n";
   ret := ret ^ ownBoard.printParallelBoards(enemyBoard);
   return ret;
  );
   
  -- mensagem de abater um navio
(*@
\label{printTakeDown:120}
@*)
 public printTakeDown : Board`CellContent ==> string
  printTakeDown(shipDown) == return "\n\n" ^ enemyBoard.shipToString(shipDown) ^ " is sinking\n";
  
  -- mensagem de vit�ria
(*@
\label{printVictory:124}
@*)
 public printVictory : () ==> string
  printVictory() == return "\nEnemy fleet destroyed. Victory!\n";

end Player
\end{vdmpp}
\bigskip
\begin{longtable}{|l|r|r|r|}
\hline
Function or operation & Line & Coverage & Calls \\
\hline
\hline
\hyperref[Player:17]{Player} & 17&100.0\% & 38 \\
\hline
\hyperref[addBoards:29]{addBoards} & 29&100.0\% & 16 \\
\hline
\hyperref[allShipsPlaced:46]{allShipsPlaced} & 46&100.0\% & 52 \\
\hline
\hyperref[clearData:57]{clearData} & 57&100.0\% & 6 \\
\hline
\hyperref[getName:25]{getName} & 25&100.0\% & 490 \\
\hline
\hyperref[printGameStatus:108]{printGameStatus} & 108&100.0\% & 40 \\
\hline
\hyperref[printInfo:96]{printInfo} & 96&100.0\% & 8 \\
\hline
\hyperref[printPlacementStatus:102]{printPlacementStatus} & 102&100.0\% & 48 \\
\hline
\hyperref[printTakeDown:120]{printTakeDown} & 120&100.0\% & 10 \\
\hline
\hyperref[printVictory:124]{printVictory} & 124&100.0\% & 2 \\
\hline
\hyperref[registerAttack:66]{registerAttack} & 66&100.0\% & 78 \\
\hline
\hyperref[registerResult:79]{registerResult} & 79&100.0\% & 54 \\
\hline
\hyperref[shipPlacement:38]{shipPlacement} & 38&100.0\% & 60 \\
\hline
\hyperref[startRounds:50]{startRounds} & 50&100.0\% & 8 \\
\hline
\hline
Player.vdmpp & & 100.0\% & 910 \\
\hline
\end{longtable}

